%% LyX 2.0.8.1 created this file.  For more info, see http://www.lyx.org/.
%% Do not edit unless you really know what you are doing.
\documentclass[english]{article}
\usepackage[T1]{fontenc}
\usepackage[latin9]{inputenc}
\usepackage{amsmath}
\usepackage{babel}
\begin{document}
A graphon is a symmetric measurable function $W:[0,1]^{2}\to[0,1]$,
that is important in the study of \textbf{dense graphs}. Graphons
arise as the fundamental objects in two areas: as the defining objects
of exchangeable random graph models and as a natural notion of limit
for sequences of dense graphs. Graphons are tied to dense graphs by
the following pair of observations: the random graph models defined
by graphons give rise to dense graphs \textbf{almost surely}, and,
by the \textbf{regularity lemma}, graphons capture the structure of
arbitrary large dense graphs.

Graphons are sometimes referred to as ``continuous graphs'', but
this is bad practice because there are many distinct objects that
this label might be applied to. In particular, there are generalizations
of graphons to the sparse graph regime that could just as well be
called ``continuous graphs.''


\section{Definition}

A graphon is a measurable function $W:[0,1]^{2}\to[0,1]$. Usually
a graphon is understood as defining an exchangeable random graph model
according to the following scheme:
\begin{enumerate}
\item Each vertex $j$ of the graph is assigned an independent random value
$u_{j}\sim U[0,1]$
\item Edge $e_{ij}$ is independently included in the graph with probability
$W(u_{i},u_{j})$.
\end{enumerate}
A random graph model is an exchangeable random graph model if and
only if it can be defined in terms of a (possibly random) graphon
in this way.

It is an immediate consequence of this definition and the law of large
numbers that, if $W\neq0$, exchangeable random graph models are dense
almost surely. \textbf{(todo: ref OR)}


\section{Examples}

The simplest example of a graphon is $W=p$ for some constant $p\in[0,1]$.
In this case the associated exchangeable random graph model is the
\textbf{Erdos-Renyi} model that includes each edge independently with
probability $p$.

The \textbf{Erdos-Renyi} model can be generalized by:
\begin{enumerate}
\item Divide the unit square into $K\times K$ block
\item Let $W$ equal $p_{lm}$ on the $l,m$th block.
\end{enumerate}
Effectively, this gives rise to the random graph model that has $K$
distinct \textbf{Erdos-Renyi} graphs with parameters $p_{ll}$ respectively
and bigraphs between them where each possible edge between blocks
$l,l$ and $m,m$ is included independently with probability $p_{lm}$.
This is simply the $K$ community \textbf{stochastic block model}.

Many other popular random graph models can be understood as exchangeable
random graph models defined by some graphon, a detailed survey is
included in \textbf{OR}.


\section{Jointly exchangeable adjacency matrices}

A random graph of size $n$ can be represented as a random $n\times n$
adjacency matrix. In order to impose consistency (in the sense of
\textbf{projectivity}) between random graphs of different sizes it
is natural to study the sequence of adjacency matrices arising as
the upper-left $n\times n$ sub-matrices of some infinite array of
random variables; this allows us to generate $G_{n}$ by adding a
node to $G_{n-1}$ and sampling edges $e_{jn}$ for $j<n$. With this
perspective, random graphs are defined as random infinite symmetric
arrays $(X_{ij})$. 

Following the fundamental importance of \textbf{exchangeable sequences}
in classical probability, it is natural to look for an analogous notion
in the random graph setting. One such notion is given by jointly exchangeable
matrices; i.e. random matrices satisfying
\begin{align*}
(X_{ij}) & \overset{d}{=}(X_{\sigma(i)\sigma(j)})
\end{align*}
for all permutations $\sigma$ of the natural numbers, where $\overset{d}{=}$
means \textbf{equal in distribution}. Intuitively, this condition
means that the distribution of the random graph is unchanged by a
relabeling of its vertices: that is, the labels of the vertices carry
no information. 

There is a representation theorem for jointly exchangeable random
adjacency matrices, analogous to de Finetti's representation theorem
for exchangeable sequences. This is a special case of the \textbf{Aldous-Hoover}
theorem for jointly exchangeable arrays and, in this setting, asserts
that the random matrix $(X_{ij})$ is generated by:
\begin{enumerate}
\item Sample $u_{j}\sim U[0,1]$ independently
\item $X_{ij}=X_{ji}=1$ independently at random with probability $W(u_{i},u_{j}),$
\end{enumerate}
where $W:[0,1]^{2}\to[0,1]$ is a (possibly random) graphon. That
is, a random graph model has a jointly exchangeable adjacency matrix
if and only if it is a jointly exchangeable random graph model defined
in terms of some graphon.


\section{Limits of sequences of dense graphs}

Consider a sequence of graphs $(G_{n})$ where the number of vertices
of $G_{n}$ goes to infinity. It is possible to define several notions
of convergence of such sequences, each of which may give rise to a
distinct limit object. For example, if the sequence $(G_{n})$ was
a realization of an \textbf{Erdos-Renyi} graphs with parameter $p$
the natural notion of limit would be the edge density of the graphs,
which converges to $p$. In this case it would be natural to say that
the limit of the sequence is the constant graphon $W=p$. It turns
out that for dense graphs a number of apparently distinct notions
of convergence are equivalent and under all of them the natural limit
object is a graphon. \textbf{LOV}

For instance, yadda yadda yadda


\section{Related notions}

Graphons are naturally associated with dense simple graphs. There
are straightforward extensions to dense directed weighted graphs \textbf{OR}.
There are also recent extensions to the sparse graph regime, from
both the perspective of random graph models \textbf{VR} and graph
limit theory \textbf{BCetal}.
\end{document}
